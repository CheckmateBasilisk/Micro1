%\documentclass[a4paper, 12pt]{report}
\documentclass[12pt,a4paper,openany]{abntex2}

\usepackage[T1]{fontenc}
%\usepackage[latin1]{inputenc}
%\usepackage[verbose,left=30mm,right=20mm,top=30mm,bottom=30mm]{geometry}
%\usepackage{txfonts}
\usepackage[brazil]{babel}
\usepackage{pdfpages}
%\usepackage[authoryear]{natbib}
%\usepackage{appendix}
%\usepackage{setspace}
%\usepackage{url}
%\usepackage{hyperref}
%\usepackage{color}
\usepackage[utf8]{inputenc}
\usepackage{placeins}
\usepackage{float}

\autor{Leonardo Mendonça de Araujo \and \\ Lucas Bagatini do Nascimento \and \\ Mário Muramatsu Júnior}
\titulo{RELATÓRIO DE FINAL: IDENTIFICADOR DE SINAIS TRIFÁSICOS}
\data{2018} 
\local{Rio Claro, São Paulo}
\preambulo{Monografia apresentada ao curso de Ciências da Computação, como requisito parcial para a obtenção do Título de Bacharel em Ciências da Computação, Instituto de Geociências e Ciências Exatas da Universidade Estadual Paulista}
\orientador{Mario Roberto da Silva}
\tipotrabalho{monografia}

\begin{document}
	
\imprimircapa	
\imprimirfolhaderosto

\clearpage
\cleardoublepage
\cleardoublepage

\pagenumbering{arabic}
\setcounter{page}{3}

\tableofcontents
\clearpage{\pagestyle{empty}\cleardoublepage}

\listoffigures
\clearpage{\pagestyle{empty}\cleardoublepage}
	
\chapter{Introdução}

\section{Sistemas Elétricos Polifásicos}

Naturalmente, sistemas elétricos funcionando em corrente alternada possuem picos e vales de tensão e, consequentemente, picos e vales de potência. A fim de mitigar esta ineficiência intrínseca, pode-se montar um sistema polifásico, ou seja, um sistema que fornece energia elétrica através de três ou mais condutores energizados.

As figuras ~\ref{fig:sinal-bifasico} e ~\ref{fig:sinal-trifasico} ilustram o comportamente da tensão em sistemas bifásicos e trifásicos, respectivamente. É trivial perceber que sistemas polifásicos são capazes de fornecer tensão e potência de maneira mais constante e sem necessitar de altos picos em cada uma de suas fases.

\begin{figure}[!htp]
	\centering
	\fbox{\includegraphics[width=16cm]{./images/Bifasico.png}}
	\caption{Exemplo de sistema Bifásico}
	\label{fig:sinal-bifasico}
\end{figure}

\begin{figure}[!htp]
	\centering
	\fbox{\includegraphics[width=16cm]{./images/Trifasico.png}}
	\caption{Exemplo de sistema Trifásico}
	\label{fig:sinal-trifasico}
\end{figure}

Um sinal trifásico, como da Figura ~\ref{fig:sinal-trifasico}, assim como outras variações de sinais polifásicos, são criados nos geradores; originados dos motores de indução polifásicos desenvolvidos concomitantemente por Galileo Ferraris, Nikola Tesla e Mikhail Dolivo-Dobrovolsky. Um gerador alternador trifásico é composto por um núcleo magnético assim como qualquer outro alternador, e seu estator é composto de bobinas ligadas aos três condutores ao invés de dois, como em um alternador simples. Conforme o rotor gira, este excita os elétrons livres nas bobinas dentro de seu campo magnético, criando corrente alternada. Existe, portanto, uma relação de proporcionalidade direta entre o espaçamento das bobinas e às distâncias das fases em cada um dos condutores.

\begin{figure}[!htp]
	\centering
	\fbox{\includegraphics[width=16cm]{./images/EsquemaTrifasico.png}}
	\caption{Esquema de um Alternador Trifásico}
	\label{fig:alternador-trifasico}
\end{figure}


https://youtu.be/4oRT7PoXSS0?t=345  em 5:45, acessado em 25/10/2018, 20:40

É possível, dada uma instalação trifásica, utilizar somente uma das fases em associação com o neutro para alimentar pequenos aparelhos, como lâmpadas e telas. Portanto, não é necessário ter duas redes elétricas dentro de uma mesma instalação industrial, por exemplo. Geralmente é desta maneira que energia elétrica é distribuída para residências, já que não espera-se que uma família tenha um forno industrial em sua cozinha.

\section{Sistemas Trifásico: Usos e Vantagens}

Como discutido acima, um sistema trifásico de transmissão (que se utiliza de três condutores) de energia possui diversas vantagens sobre sistemas monofásicos (que usam dois condutores). Dentre eles, o mais notório é o ganho em custo-benefício de tal sistema. Pode-se transmitir o triplo da energia com somente 50\% de incremento nos custos. A partir de 3 fases o incremento no custo é grande demais para compensar os benefícios.

Sistemas polifásicos são especialmente úteis e convenientes na construção e operação de motores elétricos, tal qual os motores de indução, por serem capazes de produzirem campos magnéticos rotativos de intensidades mais constantes e confiáveis. É possível alterar a frequência das três fases para alterar a velociade de rotação do motor, por exemplo. Motores de indução que utilizam corrente contínua requerem diversas peças adicionais, como transformadores, que encarecem o mecanismo final. Menores flutuações na potência entregue aos eixos também significa que menor stress será causado ao material, permitindo o uso de materiais mais baratos ou leves, aumentando ainda mais a eficiência e o custo-benefício.

De maneira análoga à motores de indução, sistemas de distribuição trifásicos são capazes de aproveitar mais da energia gerada por um gerador, como turbinas de uma hidrelétrica.

\section{Projeto: Identificador de Sinais Trifásicos}

Ao longo de grandes instalações elétricas, ao se utilizar modelos trifásicos, deve-se garantir que os diversos dispositivos e conexões estão ligados às fases adequadas. Erros desta natureza podem causar desde ineficiência do maquinário ligado até catastróficos acidentes. A fim de auxiliar na tarefa, propõe-se que seja construído um Identificador de Fase para Sinais Trifásicos.

Sabendo-se que dois condutores fazem parte de um mesmo sistema de transmissão trifásico ou rede de transmissão, o aparato deve ser capaz de fazer a leitura da frequência presente em um e compará-la à frequência do outro, indicando se há ou não defasagem entre ambos. Deve-se construir também um sistema que garanta a confiabilidade da leitura, indicando ao usuário quando o sinal lido anteriormente não é mais confiável. O aparato deve, também, mitigar variações momenâneas causadas pelo início da leitura, tal qual o repetido conecta-desconecta causado pelos imperceptíveis e repetidos choques, natural do contato inicial entre materiais duros, como o metal dos fios e a p.

O usuário, após acoplar o aparato a condutor, terá consigo uma amostra da fase obtido através de contadores internos. Ao comparar o conteúdo do aparato com a fase de outro condutor, este indicará se há adianto ou atraso de um terço de um período dosegundo em relação ao primeiro. Parte-se do pressuposto que ambos têm frequências iguais, já que são leituras em partes distintas de uma mesma rede ou linha de transmissão.

Pode-se utilizar o aparato para garantir a sincronia entre duas linhas de transmissão distindas, regulando a frequência e fase da corrente alternada que passa por uma delas.

\chapter{Materiais e Metodologia}

Para a etapa de projeto conceitual e virtual dos módulos do circuito foi utilizado
o software Quartus na sua versão 9.0 com licença universitária.
Constituindo as funcionalidades do Quartus temos a área de trabalho do projeto,
em que podem ser arrastados elementos de hardware como portas lógicas, pinos de
entrada e saída, flip-flops, fios, barramentos e muito mais.
Além da área de trabalho temos a opção de fazer alterações nos componentes do
circuito por meio do uso de HDL (Linguagem de Descrição de Hardware), e também é
possível executar testes funcionais e/ou de tempo para verificar a integridade dos
módulos desenvolvidos antes de testá-los na placa física.
A Figura ~\ref{fig:interface-quartus} mostra a interface do usuário do Quartus versão 9.0.

%Figura X (interface-quartus)
\begin{figure}[!htp]
	\centering
	\fbox{\includegraphics[width=16cm]{images/interface-quartus.png}}
	\caption{Interface do Programa Quartus}
	\label{fig:interface-quartus}
\end{figure}

\section{NOME DA SEÇÃO}

Juntamente com o software Quartus versão 9.0, foi usada para o desenvolvimento e teste
de todas as etapas do projeto a placa UP Educational Board, que está representada
detalhadamente pela Figura ~\ref{fig:placa-1}.

%Imagem da placa_1
\begin{figure}[!htp]
	\centering
	\fbox{\includegraphics[width=16cm]{images/placa_1.png}}
	\caption{Overview da Placa}
	\label{fig:placa-1}
\end{figure}

Alguns dos elementos da placa usados para a simulação do funcionamento dos módulos
foram os pinos de entrada e saídas de led para identificar saídas do circuito.

* Explicar necessidade do Antibouncing
\section{NOME DA SEÇÃO}

Quando trabalhamos com um circuito que envolve entradas mecânicas (como a ponta de prova),
tem-se o surgimento de um efeito conhecido como “bouncing”.
Esse efeito é caracterizado por “idas” e “vindas”, do nível lógico alto e o nível
lógico baixo, antes de efetivamente estabilizar. Essas oscilações rápidas podem
gerar acionamentos indevidos no nosso circuito. Pois o mesmo tende a interpretar
que a ponta de prova foi acionada repetidas vezes em um curto espaço de tempo,
quando na verdade foi apenas uma vez.
Devido a isso, foi necessário o desenvolvimento de um módulo de Antibouncing
integrado ao sistema para lidar com o problema. Usando técnicas de desenvolvimento
de máquinas sequenciais conseguimos que o sinal de entrada similar ao da imagem A
fosse manipulado, gerando uma saída como o apresentado na imagem B.

%Imagem A (sinal com bouncing)
\begin{figure}[!htp]
	\centering
	\fbox{\includegraphics[width=16cm]{images/bounce_on.png}}
	\caption{Sinal com Bouncing}
	\label{fig:sinal-com-bouncing}
\end{figure}

%//Imagem B (sinal sem bouncing)
\begin{figure}[!htp]
	\centering
	\fbox{\includegraphics[width=16cm]{images/bounce_off.png}}
	\caption{Sinal sem Bouncing}
	\label{fig:sinal-sem-bouncing}
\end{figure}

* Explicar necessidade do Timeout
\section{NOME DA SEÇÃO}

Após ser implementado o módulo de AntiBouncing, o grupo se deparou com uma nova
situação problemática, foi identificado que o sinal capturado pela ponta de prova
(podendo ser R, S ou T), como já era esperado, começava a se defasar com o passar
dos segundos, no entanto, depois de um período específico essa defasagem tornava-se
tão severa que o sinal já não podia mais ser identificado como R,S ou T.
Esse período foi chamado de período de Timeout do sinal obtido pela ponta de prova
e foi necessário o desenvolvimento de um módulo adicional integrado ao sistema que
fosse capaz de cronometrar esse período e que identificasse quando um sinal contido
na entrada da ponta de prova seria confiável ou não.
Se identificado o sinal não-confiável seria necessário a obtenção de um novo sinal
por meio do acionamento da ponta de prova, tendo novamente um sinal aceitável para
ser manipulado novamente pelos outros módulos do circuito.

\chapter{Discussão e Atividades}

\section{Emulador RST}

O primeiro passo para começarmos a desenvolver o projeto era criar uma representação do sinal trifásico, para que este sirva de referência para os testes futuros.

A principio, sabemos que os sinais trifásicos possuem defasagem de 120º entre cada onda e oscilam a 60Hz, sendo assim, é necessário criar uma máquina sequencial que seja capaz de gerar tal sinal.

\begin{figure}[!htp]
	\centering
	\caption{Sinais RST}
	\fbox{\includegraphics[width=16cm]{images/sinalRST}}
	\label{fig:sinais-rst}
\end{figure}

O primeiro passo foi elaborar um diagrama de sequência que satisfaça as condições para a geração dos sinais.

\begin{figure}[!htp]
	\centering
	\caption{Diagrama de Estado Emulador RST}
	\fbox{\includegraphics[width=16cm]{images/disgrama-de-estado-EmuladorRST}}
	\label{fig:disgrama-de-estado-EmuladorRST}
\end{figure}

Para a construção deste diagrama foi utilizado três bits para referenciar os estados, note que de um estado para outro apenas um bit é alterado, foi feita esta escolha pelo fato de que com isso a implementação do circuito iria ser simplificada. Montando a tabela de próximos estados e de excitação dos flip-flops.

\begin{figure}[!htp]
	\centering
	\caption{Tabela de estados e excitação dos Flip-flops Emulador RST}
	\includegraphics[width=16cm]{images/tabela-RST}
	\label{fig:tabela-RST}
\end{figure}

Utilizando Karnaugh para fazer a simplificações e encontrar as expressões de cada saída do circuito, encontramos que:

$$DR = Q1$$
$$DS = \overline{Q0}$$
$$DT = Q2$$

As saídas serão:

$$R = \overline{Q2}$$
$$S = Q1$$
$$T = Q0$$

A partir disto podemos construir o circuito do emulador RST, representado pela Figura \ref{fig:circuito-RST}.

\begin{figure}[!htp]
	\centering
	\caption{Circuito do Emulador RST}
	\includegraphics[width=16cm]{images/circuito-RST}
	\label{fig:circuito-RST}
\end{figure}

Tal circuito está gerando sinais periódicos defasados em 120º, porém apenas com isto não é possível que elas oscilem a 60Hz, pois a placa em que foram feito os teste possuí clock de 25.175MHz, portanto, é necessário reduzir o clock de entrada de tal circuito. 

Para fazer tal redução foi utilizado a saída Carry-out de um contador de módulo 34965 como o clock do nosso circuito RST, com isso o clock será dividido pelo módulo do contador. Mas isso ainda não é suficiente para alcançar os 60Hz, para atingir o valor correto foi inserido um flip-flop do tipo T para dividir o clock por dois, fazendo com que entre no circuito um sinal de 360Hz que será dividido por seis ao passar pelo emulador RST, devido ao fato desta ter 6 estados. Assim teremos os sinais RST a 60Hz.

A Figura \ref{fig:circuito-RST-complete} representa o circuito que gerará os sinais corretamente.

\begin{figure}[!htp]
	\centering
	\caption{Circuito do Emulador RST com saídas a 60Hz}
	\includegraphics[width=16cm]{images/circuito-RST-complete}
	\label{fig:circuito-RST-complete}
\end{figure}

\section{Medidor de Período}

Primeiramente precisamos ser capazes de receber um sinal de entrada e interpreta-lo para que seja possível dizer qual das três fases foi inserida, então é necessário criar um circuito capaz de identificar qual é o período da fase inserida, representado pela Figura \ref{fig:diagrama-medidor-de-frequencia}.

\begin{figure}[!htp]
	\centering
	\caption{Diagrama esquemático do medidor de período da fase de referência}
	\includegraphics[width=16cm]{images/diagrama-medidor-de-frequencia}
	\label{fig:diagrama-medidor-de-frequencia}
\end{figure}

Sabendo que o período é toda a extensão desde sua primeira
borda de subida até a sua próxima borda de subida, então, para conseguir identifica-lo, foi utilizado um contador que irá contar o número de clocks dentro de um período e então armazena-lo em um registrador.

Ao inicio de cada período o contador é zerado através do sinal Cl da unidade de controle UCf, tal circuito será explicada mais a frente. no mesmo clock, através do sinal Ld o ultimo valor contado é transferido para o registrador, para que ambos comandos sejam executados em ordem correta foi criado a unidade X2.

\begin{figure}[!htp]
	\centering
	\caption{Unidade X2}
	\includegraphics[width=16cm]{images/unidade-x2}
	\label{fig:unidade-x2}
\end{figure}

A única função da unidade da Figura \ref{fig:unidade-x2} é, a partir de uma entrada, no caso o clock, gerar duas saídas, uma idêntica ao sinal de entrada e a outra com o sinal negado.

Note que, o clock do contador provem do sinal CK2 de X2, ou seja, tal contador trabalha na borda de descida do clock original, devido a isto os comandos Cl e Ld podem ser feitos no mesmo clock, uma vez que, assim que forem gerados sinais altos de Cl e Ld, na borda de subida do clock, o valor contado pelo contador é salvo no registrador e no borda de descida deste mesmo clock, é feito o clear do contador.

\section{Correção de Erro aleatório}

Utilizando a unidade X2 para sincronizar os comandos faz com que exista a geração de um erro aleatório, possuindo dois casos extremos.

\begin{itemize}	
	\item Exite o caso em que o sinal de entrada é inserido exatamente antes da borda de subida do clock, então a zeragem do contador possuirá um atraso de 0,5 clock, ou seja, um erro de -0,5 clock, porém note que no valor registrado é desconsiderado 0,5 clock, com isto é gerado mais um erro de -0,5 clock, mostrado na Figura \ref{fig:erro-leitura1};
\end{itemize}


\begin{itemize}	
	\item No caso em que o sinal de entrada é inserido exatamente depois da borda de subida do clock, o contador será zerado com um atraso de 1,5 clocks, ou seja, com um erro de -1,5 clocks. Além disto o valor é registrado com 1 clock a mais, gerando um erro de +0,5 clock na leitura, representado pela Figura \ref{fig:erro-leitura2}.
\end{itemize}

\begin{figure}[!htp]
	\centering
	\caption{Entrada exatamente antes da borda de subida}
	\includegraphics[width=7cm,height=7cm]{images/erro-leitura1}
	\label{fig:erro-leitura1}
\end{figure}

\begin{figure}[!htp]
	\centering
	\caption{Entrada exatamente depois da borda de subida}
	\includegraphics[width=7cm,height=7cm]{images/erro-leitura2}
	\label{fig:erro-leitura2}
\end{figure}

Analisando todas as combinações de erros possíveis nos dois casos extremos e utilizando a operação de convolução para estimar a probabilidade conjunta do dois eventos (carga e zeragem), entao, pode-se concluir que há a geração de um erro aleatório do qual varia de acordo com uma distribuição de probabilidade contínua com centro em -1 e bordas em -2 e 0. A Figura \ref{fig:distribuicao-de-probabilidade-do-erro} representa tal distribuição.

\begin{figure}[!htp]
	\centering
	\caption{Distribuição de probabilidade do erro}
	\includegraphics[width=14cm,height=10cm]{images/distribuicao-de-probabilidade-do-erro}
	\label{fig:distribuicao-de-probabilidade-do-erro}
\end{figure}

Note que, pelo fato da distribuição ter centro em -1, então o erro mais provável é o de -1 clock, ou seja, a cada leitura será perdido um clock da fase de entrada, fazendo com que as medições utilizem dados incorretos. O ideal nesta situação é que o centro da curva de distribuição de probabilidade seja exatamente em 0, assim o erro mais provável será de 0 clocks, tendo assim uma leitura mais correta da fase de entrada.

No primeiro circuito, o contador possuía uma entrada sclr (Synchronous Clear), este comando permitia zerar todos os bits do contador, porém, para corrigir tal problema alterou-se esta entrada para um sset (Synchronous set), este comando irá inserir um valor preestabelecido, no caso, será inserido o valor 1.

Desta forma iremos inserir um erro de +1 clock na leitura, uma vez que o primeiro clock não será contado, e sim inserido, porém, isso faz com que a distribuição de probabilidade atinja o comportamento ideal, ou seja, somando 1 ao erro aleatório o centro da curva ficará em 0, representado pela Figura \ref{fig:distribuicao-de-probabilidade-do-erro1}.

\begin{figure}[!htp]
	\centering
	\caption{Distribuição de probabilidade do erro com centro em 0}
	\includegraphics[width=8cm,height=5cm]{images/distribuicao-de-probabilidade-do-erro1}
	\label{fig:distribuicao-de-probabilidade-do-erro1}
\end{figure}

Note que esta solução não corrige todo o problema apenas diminui a probabilidade de erros, o ideal seria que não houvessem erros na leitura. Para que seja possível garantir que o sinal lido é confiável foi criado um outro circuito que visa promover isto, tal circuito será apresentado mas a frente.

\section{Gerador}

O circuito atual é capaz de medir o período de um sinal de entrada, porém, tudo será perdido se o sinal for retirado, o próximo passo então é criar um circuito capaz de gerar um sinal em sincronismo com o sinal de referência recebido. 

Primeiramente foi construído um diagrama de estado para uma UC (Unidade de Controle) capaz de gerenciar os componentes do circuito com o objetivo de fazer a cópia do sinal recebido, representada pela Figura \ref{fig:diagrama-de-estado1}.

\begin{figure}[!htp]
	\centering
	\caption{Diagrama de estados da nova UC}
	\fbox{\includegraphics[width=16cm]{images/diagrama-de-estado1}}
	\label{fig:diagrama-de-estado1}
\end{figure}

Ao conectar um sinal de entrada ao circuito, este começará no estado 1 e então irá para o estado 2 quando PP (sinal de entrada) estiver em 1. Para passar para o estado 3 é necessário que PP esteja em 0. Note que este processo é feito para ignorar o primeiro período já que ele pode não ser um sinal muito confiável, devido ao fato de não sabermos se o sinal lido inicialmente está exatamente no inicio de um período da fase.

Chegando ao estado 4, o sinal Cl ficará em nível alto, limpado assim o contador, em seguida passa-se para o estado 5, em que, junto com o estado 6 irão esperar passar outro período, chegando finalmente no estado 7, em que são gerados sinais Cl e Ld. Garantida esta primeira medição correta do período, a máquina evolui sucessiva e repetidamente pelos estados 8, 9 e de volta para o 7, onde são providos os sinais Cl e Ld, permitindo a realização de novas e corretas medições, até que o sinal de entrada for retirado e o sinal Comp esteja em 1.

O sinal Comp irá ficar em 1 sempre que o contador atingir o valor lido nos estado 4, 5 e 6, ou seja, quando completar um período da fase de entrada.

Indo para o estado 10, onde é gerado um sinal Cl para manter o sincronismo com a fase lida e então passa para o estado 1. A partir deste ponto o a maquina fica repetidamente indo do estado 1 para o 10, esta transição irá ocorrer sempre que o sinal de entrada estiver em 0 e o Comp estiver em alto, ou seja, sempre ao fim de um período.

A partir deste diagrama de estado podemos construir o circuito da Figura \ref{fig:gerador}. Note que parte do circuito foi reaproveitado, o contador e o registrador. O comparador abaixo (inst 11) faz a comparação entre o sinal corrente e o valor de frequência armazenado, gerando assim o sinal Comp que é utilizado pela UC. Tal sinal, possui um ciclo ativo muito pequeno, que impede a visualização deste em um osciloscópio, a fim de gerar um sinal capaz de ser visualizado foi inserido o comparador na parte superior (inst 10), que irá comparar a contagem em curso com a metade da contagem registrada que será obtido do registrador de deslocamento ao lado (inst 5), fazendo com que o sinal "copia" de saída possua ciclo ativo de 50\% síncrono com o sinal de entrada.

Outra mudança necessário foi a troca do registrador (inst9) para um que começasse com o valor máximo possível, ou seja, todos os bits em 1. Isto se deve pois, como o lpm\_counter1 (inst 8) começa sempre em 1, e se o comparador começar em 0, o sinal Comp não será ativado corretamente.

\begin{figure}[!htp]
	\centering
	\caption{Diagrama de estados da nova UC}
	\includegraphics[width=16cm,height=7cm]{images/gerador}
	\label{fig:gerador}
\end{figure}

\section{Comparador de Fase}

Após inserir um sinal de entrada, o circuito é capaz de gerar uma cópia desta fase, porém, é impossível descobrir qual fase foi inserida (R, S ou T), para isso, foi implementado três botões (Rin, Sin e Tin) que devem ser pressionados para indicar qual fase foi inserida inicialmente.

Ao pressionar um dos botões de entrada, a saída LedOutput deve ser equivalente ao botão pressionado, isso acontece até que se tire o dedo do botão. Assim quando chegar um novo sinal de Clear, não existirá mais sinal desses botão, contudo a saída deve continuar a mesma, isto porque a lógica do componente saidaLed é realimentada com os valores de Rout, Sout e Tout. Apenas terá alteração na saída ao colocar um novo sinal na ponta de prova.

\subsection{Máquina sequencial CompFase}

Neste ponto, a máquina está pronta para executar a leitura de uma nova fase para fazer a comparação. A máquina sequencial da Figura \ref{fig:comp-fase} sinaliza se o novo sinal de entrada está igual a fase de referencia, se é a fase anterior (na sequencia RST) ou a fase posterior, para isto utiliza a seguinte lógica:
\begin{itemize}
	\item estados começando em 00, indicam que a fase inserida é igual a fase de referencia;
	\item estados começando em 01, indicam que a fase inserida é a posterior à fase de referencia;
	\item estados começando em 10, indicam que a fase inserida é a anterior à fase de referencia.
\end{itemize}

Perceba que o ultimo bit serve para diferenciar os estados cujos valores de saídas são iguais.

\begin{figure}[!htp]
	\centering
	\caption{Diagrama de estados da comparadora de fases}
	\fbox{\includegraphics[width=16cm,]{images/comp-fase}}
	\label{fig:comp-fase}
\end{figure}

\subsection{Circuito CompFase}

A partir da máquina sequencial apresentada, pode-se construir o circuito da Figura \ref{fig:circuito-compfase}.

\begin{figure}[!htp]
	\centering
	\caption{Circuito CompFase}
	\includegraphics[width=16cm]{images/circuito-compfase}	\label{fig:circuito-compfase}
\end{figure}

Para utilizar o método de comparação de fases requer uma frequência de seis vezes a frequência da fase de referência. Assim, o contador Ipm\_counter3 (inst) gera esta frequência dividindo o clock por seis, sendo então utilizado para realizar uma contagem entre as bordas de subida da fase de referência pelo contador Ipm\_counter5 (inst2), resultando em uma contagem truncada em $\frac{1}{6}$ do período da fase de referência, sendo armazenada em um registrador Ipm\_dff1 (inst3) pelo pulso Ld.

Como o contador Ipm\_counter5 (inst2) recebe um clock com $\frac{1}{6}$ da
frequência original do circuito, é utilizado um Clear assíncrono que ocorre na borda de descida anterior ao pulso Cl, com metade de sua duração, proveniente da porta "E" AND2 (inst10) entre o pulso Cl e o flip-flop tipo D DFF (inst9), que também realiza o Clear na máquina comparadora de fases.

O comparador Ipm\_compare (inst4) provém então o clock necessário para a
máquina comparadora de fases, e realiza também o Clear do contador Ipm\_counter4 (inst1), através da comparação entre a contagem truncada em $\frac{1}{6}$ do período da fase de referência e a do próprio contador, após o atraso de meio período de clock, obtido com o flip-flop tipo D DFF (inst9), pois sua saída também ocorre em uma borda de descida. Como uma divisão por seis pode envolver resto, o contador Ipm\_counter4 (inst1) também recebe outro sinal de Clear através da porta "OU" OR2 (inst5) produzido a cada período da fase de referência, evitando-se o acúmulo de erros.

\section{Timeout}

Na seção 3.3, em que corrigimos o problema de erro aleatório na leitura da fase de referência, foi dito que aquilo era o suficiente para acabar com os erros de leitura, uma vez que, os erros variam de acordo com uma distribuição de probabilidade continua com centro em 0, isto é, pode variar de -1 a +1 clock de erro.

Este valor pode parecer pequeno, mas este erro se repete a cada novo período enquanto se espera a conexão da nova fase de referência à ponta de prova. O sinal gerado internamente se defasa em relação a fase de referência para mais ou para menos.

Se o acumulo deste erro se torna muito grande então a comparação pode não ser tão precisa, pois, no caso em que a fase R é inserida como fase de referencia, após a cópia se defasar em $\frac{1}{6}$ para menos em relação a fase de referencia e inserindo novamente uma fase R para fazer a comparação, tal situação se confunde com o caso em que é inserido a fase S para fazer a comparação e a cópia se defasou em $\frac{1}{6}$ para mais.

Para corrigir isto criou-se o circuito Timeout, que nada mais é do que um "temporizador". Este circuito ira executar uma contagem decrescente, até que a fase de referencia esteja defasada (situação limite é quando esta atinge $\frac{1}{6}$ de erro). Ao atingir o valor 0, a máquina será travada, indicando que o sinal copiado não é mais confiável, necessitando de uma nova leitura para a fase de referencia.

O tempo deste Timeout pode ser estimado, sendo sua fórmula do tipo: $$Timeout = \frac{Clock}{\left(\left. 6 \right. . \left. \left(frequencia\right)^2\right.\right)}$$

Sendo o clock da placa igual a $25175$MHz e a frequência da rede entre $60$Hz a $70$Hz, temos o tempo aproximadamente de $856s$, equivalente a 14 minutos e 16 segundos.

\subsection{Circuito contador}

Para fazer a contagem, construiu o circuito da Figura \ref{fig:timeout}.
O contador lpm\_counter5 (inst8) irá dividir o clock de $25175$MHz por dois, gerando um sinal de $2$Hz que será divido novamente por 2 pelo flip-flop tipo T gerando um sinal de $1$Hz. Tal sinal é então dividido pelo contador de módulo 60 (inst3) produzindo um pulso a cada minuto. Os contadores à direita irão receber estes sinal, sendo que os dois superiores executam a contagem em segundos e os inferiores a contagem em minutos, o cout dos contadores de unidades servem para acionar os contadores de dezenas.

O valor inicial calculado acima é inserido através dos pinos aset (Asynchronous Set) dos contadores.

\begin{figure}[!htp]
	\centering
	\caption{Circuito de contagem do timeout}
	\includegraphics[height=13cm]{images/timeout}
	\label{fig:timeout}
\end{figure}

Quando tais contadores de tempo em minutos alcança 00 é produzido o sinal que deverá travar a máquina toda até que nova fase seja colocada na ponta de prova e esta seja identificada pelo pressionar de um dos botões.

\subsection{Circuito apresentador do Timeout}

Ao atingir o valor de 2 minutos, o contador de segundo deverá estar com tempo restante de 20 segundos, após passar este tempo, a contagem passará a mostrar apenas o valor em segundos, sendo o que este começará em 99 segundos. A combinação entre o valor 2 dos minutos com o 9 dos segundos gera o sinal MIN\_SEC que irá alterar a contagem apenas para segundos. O circuito que faz tão função é representado pela Figura \ref{fig:display-timeout}

\begin{figure}[!htp]
	\centering
	\caption{Circuito do display do Timeout}
	\includegraphics[width=16cm]{images/display-timeout}	\label{fig:display-timeout}
\end{figure}

O bloco lógico Time\_out é representado pelo seguinte diagrama de estado, Figura \ref{fig:diagrama-de-estado-display-timeout}. O circuito irá para o estado 3, onde a contagem de minutos esta ativa, nos estados 1 e 2, a contagem de minutos é desativada. Quando a máquina está nos estados 2 e 3, os pinos enable dos contadores estarão ativos, liberando os contadores para continuarem a contar, porém, no estado 1 estes enables são desativados, travando a contagem.

\begin{figure}[!htp]
	\centering
	\caption{Diagrama de estado do diplay do Timeout}
	\fbox{\includegraphics{images/diagrama-de-estado-display-timeout}}
	\label{fig:diagrama-de-estado-display-timeout}
\end{figure}

A transição do estado 3 para o 2 ocorre quando o sinal MIN\_SEC estiver em 1, ou seja, quando a contagem atinge 99 segundos. No estado 2, caso a ponta de prova seja conectada antes do termino do tempo de timeout, a máquina voltará para o estado 3, do contrario ele irá para o estado 1, onde irá ficar travada até que aja uma nova leitura na ponta de prova.

\section{Anti bouncing}

Ao conectar a ponta de prova ao sinal de referencia ocorre um fenômeno chamado de bouncing, que consistem em oscilações aleatórias geradas pelo contato mecânico entre o fio e a ponta de prova. Isto ocorre devido à condição física dos conectores, uma vez que estes possuem uma superfície rugosa causando uma instabilidade inicial, que provocam pequenas oscilações. Este fenômeno é representado pela Figura \ref{fig:bounceTimingDiagram}

\begin{figure}[!htp]
	\centering
	\caption{Fenômeno Anti bouncing}
	\fbox{\includegraphics[width=10cm,height=4cm]{images/bounceTimingDiagram}}
	\label{fig:bounceTimingDiagram}
\end{figure}

Tais oscilações podem influenciar nos resultados, fazendo com que a máquina sequencial transite pelos estados sem ter um sinal confiável.

Para corrigir este problema, foi proposto a seguinte máquina sequencial (Figura \ref{fig:antibouncing-diagrama}) que servirá como Unidade de controle do gerador da cópia. Basicamente ele irá travar a leitura até que se passe um tempo necessário para as entradas se estabilizarem.

\begin{figure}[!htp]
	\centering
	\caption{Diagrama de estado da Unidade Anti bouncing}
	\fbox{\includegraphics{images/antibouncing-diagrama}}
	\label{fig:antibouncing-diagrama}
\end{figure}

Ao iniciar a máquina, ela permanecerá no estado 1 até que uma ponta de prova seja conectada ao dispositivo e o tempo característico de bouncing ja tenha passado e o sinal H, que é recebido através do pressionar de um dos botões que indicam qual é a fase inserida, esteja em 1. Se estas condições forem cumpridas então a máquina evoluirá para o estado 9, onde produzirá um sinal Rf (Reset Flag), por um clock, e irá para o estado 12 caso o sinal Comp esteja em 1, do contrário irá para 10. O caso em que Comp estiver em 1 significa que houve um transcurso de mais um período registrado, para isto, é necessário então que o sinal Cl esteja em 1, Cl é colocado em nível alto, por um clock, no estado 12 e em seguida ele irá para o estado 10.

A máquina ficará no ciclo entre o estado 10 e 12 até que o sinal F esteja em 1, ou seja, até passar o tempo característico de bouncing. Os estados 2 e 3 irão permitir a detecção da próxima borda de subida da fase. No estado 4, é gerado um sinal Cl, que irá zerar o contador de período, no estado 5 e 6 seja possível fazer a contagem do período corretamente e então salvando-o no estado 7 com um sinal Ld e outro Cl, para finalmente retornar ao estado 1.

Neste ponto, sempre que o sinal Comp estiver em 1 a máquina irá para o estado 8, em que será gerado um sinal Cl para mantar o sincronismos com a fase de referencia. Caso a ponta de prova ainda esteja conectada e em nível alto, a máquina irá para o estado 11, para gerar um sinal Rf para impedir que F seja ativada, e logo em seguida irá voltar para o estado 1 ou primeiro irá para o estado 8, pela necessidade de um clear e em seguida retornará ao 1.

A partir deste diagrama foi construído o circuito da Figura \ref{fig:antibouncing-uc}.

\begin{figure}[!htp]
	\centering
	\caption{Circuito da Unidade de controle com Anti bouncing}
	\includegraphics[width=16cm]{images/antibouncing-uc}	\label{fig:antibouncing-uc}
\end{figure}

O sinal Rf que é gerado para não permitir que a entrada F esteja em 1 nada mais é do que um sclr (Synchronous clear) do lpm\_counter (inst14). Este contador irá marcar o tempo característico de bouncing, ao final desta contagem, o sinal cout (Carry out) será expelido, indicando que o contador atingiu o seu limite, ou seja, passou-se o tempo necessário. Note que esta saída influencia o próprio contador, uma vez que o sinal cout é inserido na entrada cnt\_en (Counter Enable). Ao inseri-la, o contador será desativado, até que seja dado um clear.

\chapter{Resultados e Discussão}

\subsection{Teste - Circuito RST}

O primeiro circuito implementado foi o gerador de sinais RST, pois este serviria para fazer todos os testes seguintes. Para que fosse possível visualizar a variação das ondas geradas pelo circuito, foi necessário diminuir o valor do contador de lpm\_counter2 (inst12) da Figura \ref{fig:circuito-RST-complete}, pois este estaria variando a 60Hz, algo que dificultaria a visualização. Alterando o valor para 10 chegamos aos resultados da Figura \ref{fig:teste-rst}.

\begin{figure}[!htp]
	\centering
	\caption{Simulação funcional com o emulador de sinais RST}
	\includegraphics[width=16cm]{images/teste-rst}	\label{fig:teste-rst}
\end{figure}

Assim como o esperado o circuito é capaz de gerar três ondas defasadas em 120º, podendo concluir que tal circuito apresenta uma representação próxima ou até mesmo fiel à situação real.

Este circuito foi testado na placa utilizando os displays de 7 seguimentos presentes nela. Cada sinal RST era representado por um seguimento, quando o sinal de uma das fases ficasse em nível alto, o seu respectivo seguimento era aceso. 

\subsection{Teste - Gerador e Anti bouncing}

Para conseguir teste esta parte foi necessário assumir quer o sinal H sempre estaria em 1, isto porque a parte da implementação dos botões que indicariam qual é a fase de entrada não havia sido implementada neste ponto. O sinal de entrada utilizado foi gerado pelo emulador RST e para simular o fenômeno de bouncing utilizamos um multiplexador que simularia uma instabilidade na entrada.
2777 ns
O teste foi feito com o clock em 2777ns no circuito da Figura \ref{fig:teste-antibouncing}.

\begin{figure}[!htp]
	\centering
	\caption{Circuito de testes do Gerador e Anti bouncing}
	\includegraphics[width=16cm]{images/teste-antibouncing}	\label{fig:teste-antibouncing}
\end{figure}

Note que a parte inferior do circuito é idêntica ao emulador RST, a única mudança é o multiplex inserido nas saídas deste. O pino de entrada PP foi utilizado para selecionar uma das fases que seriam retroalimentadas na Unidade com o anti bouncing.

Outra mudança necessário foi a alteração do módulo do contador lpm\_counter (inst4) da Figura \ref{fig:antibouncing-uc} para 400, pois do contrário o tempo característico de bouncing iria ser muito grande e então não seria possível ver o funcionamento do circuito.

Para a execução do teste foi inserido inicialmente o sinal R para que seja executado a cópia e depois que o sinal de referencia estivesse sincronizado, a ponta de prova foi colocada em nível baixo, e em seguida inserido o sinal S.

Utilizando o clock como 2777ns, foi gerados os resultados da Figura \ref{fig:teste-resultados-antibouncing}

\begin{figure}[!htp]
	\centering
	\caption{Simulação funcional do Gerador e Anti bouncing - Parte 1}
	\includegraphics[width=16cm]{images/teste-resultados-antibouncing}	\label{fig:teste-resultados-antibouncing}
\end{figure}

Note que assim que começamos a criar o fenômeno de bouncing no pino PP um sinal Rf é gerado e então F fica em nível baixo até passar o tempo característico de bouncing. Quando F volta para nível alto, um pulso é dado em CL, para zerar o contador e começar a contar o valor corretamente. No final deste período é produzido novamente o CL para manter a sincronia entra a fase de referencia e um Ld para salvar o valor contado. Note que a partir deste ponto o pino cópia começa a gerar um sinal idêntico à R, e assim como o esperado vário pulsos em Rf são gerados para impedir F de ir para nível alto.

\begin{figure}[!htp]
	\centering
	\caption{Simulação funcional do Gerador e Anti bouncing - Parte 2}
	\includegraphics[width=16cm]{images/teste-resultados-antibouncing2}	\label{fig:teste-resultados-antibouncing2}
\end{figure}

Também foi simulado a retida da ponta de prova, para isto também a simulação do fenômeno de bouncing, este teste está representado na Figura \ref{fig:teste-resultados-antibouncing2}. Quando começamos a gerar a instabilidade na ponta de prova, vários Rf são feitos, até que PP se torna 0 e os pulsos em Rf para, porém, o pino cópia continua a gerar um sinal em sincronia com o sinal R.

\begin{figure}[!htp]
	\centering
	\caption{Simulação funcional do Gerador e Anti bouncing - Parte 3}
	\includegraphics[width=16cm]{images/teste-resultados-antibouncing3}	\label{fig:teste-resultados-antibouncing3}
\end{figure}

Ainda neste mesmo teste, simulamos a inserção de uma nova fase de referencia, representado pela Figura \ref{fig:teste-resultados-antibouncing3}. A princípio começamos a gerar a instabilidade no sinal de PP, assim como na primeira leitura, é gerado um pulson em Rf para zerar o contador da UC e começar a contar o tempo característico de bouncing. 

O fim deste tempo acontece quando F volta para o nível alto e então é gerado um sinal Cl na primeira borda de subida de S depois do tempo de bouncing. Ao fim deste período é gerado os sinais Cl e Ld, assim como na primeira leitura. Note que antes deste do primeiro Cl depois do tempo característico de bouncing o sinal de cópia ainda era igual a R, após isto, o sinal passa a ser igual ao S.

Este circuito foi testado na placa, porém não apresentou resultados positivos. Seu comportamento, quando rodando na placa, apresentava erros dos quais não fomos capazes de descobrir. Tentamos várias modificações, porém, alguns erros persistiam, outros sumiam e novos erros surgiam, impossibilitando uma conclusão precisa.

\section{Teste - Demais componentes}

Os demais componentes não foram possíveis de serem testados, devido a dúvidas de implementação, testes sem sucesso e falta de tempo.

\chapter{Conclusão}

\end{document}
